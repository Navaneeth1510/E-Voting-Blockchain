\chapter{Introduction}
\thispagestyle{special}
\section{DATABASE TECHNOLOGIES}
The essential feature of database technology is that it provides an internal representation (model)
of the external world of interest. Examples are, the representation of a particular
date/time/flight/aircraft in an airline reservation or of the item code/item description/quantity on
hand/reorder level/reorder quantity in a stock control system.
The technology involved is concerned primarily with maintaining the internal representation
consistent with external reality; this involves the results of extensive R&D over the past 30 years
in areas such as user requirements analysis, data modelling, process modelling, data integrity,
concurrency, transactions, file organisation, indexing, rollback and recovery, persistent
programming, object-orientation, logic programming, deductive database systems, active database
systems... and in all these (and other) areas there remains much more to be done. The essential
point is that database technology is a CORE TECHNOLOGY which has links to:
\begin{itemize}
\item{Information management / processing
}
\item{Data analysis / statistics}
\item{Data visualization / presentation}
\item{Multimedia and hypermedia}
\item{Office and document systems}
\item{Business processes, workflow, CSCW (computer-supported cooperative work)}
\end{itemize}
Relational DBMS is the modern base technology for many business applications. It offers
flexibility and easy-to-use tools at the expense of ultimate performance. More recently relational
systems have started extending their facilities in directions like information retrieval, object orientation and deductive/active systems which lead to the so-called 'Extended Relational Systems'. 
\\
Information Retrieval Systems began with handling library catalogues and then extended to full
free-text by utilizing inverted index technology with a lexicon or thesaurus. Modern systems utilize
some KBS (knowledge-based systems) techniques to improve the retrieval.
Object-Oriented DBMS started for engineering applications in which objects are complex, have
versions and need to be treated as a complete entity. OODBMSs share many of the OOPL features
such as identity, inheritance, late binding, overloading and overriding. OODBMSs have found
favours in engineering and office systems but haven’t been successful yet in traditional application
areas.
Deductive / Active DBMS has evolved over the last 20 years and combines logic programming
technology with database technology. This allows the database itself to react to the external events
and also to maintain its integrity dynamically with respect to the real world.

\section{ CHARACTERISTICS OF DATABASE APPROACH}
Traditional form included organising the data in file format. DBMS was a new concept then, and
all kinds of research was done to make it overcome the deficiencies in traditional style of data
management. A modern DBMS has the following characteristics −
\begin{itemize}
        \item{Real-world entity − A modern DBMS is more realistic and uses real-world entities to design
its architecture. It uses behaviour and attribute too. For example, a school database may use
students as an entity and their age as an attribute.}
     \item{Relation-based tables − DBMS allows entities and relations to form tables.
A user can understand the architecture of a database by just looking at the table names.}
    \item{Isolation of data and application − A database system is entirely different than its data. A
database is an active entity, whereas data is said to be passive, on which the database works
and organizes. DBMS also stores metadata, which is data about data, to ease its own
process.}
    \item{Less redundancy − DBMS follows the rules of normalization, which splits a relation when
any of its attributes has redundancy in its values. Normalization is a mathematically rich
and scientific process that will reduces the data redundancy}
        \item{Consistency − Consistency is a state where every relation in a database remains consistent.
There exists methods and techniques, that can detect an attempt of leaving database in an
inconsistent state. DBMS can provide greater consistency as compared to earlier forms of
data storing applications like file-processing systems.
}
\item{Query Language − DBMS is equipped with query language, which makes it more efficient
to retrieve and manipulate data. A user can apply as many and the filtering options as
required to retrieve a set of data. Traditionally it was not possible where file-processing
system was used}
\item{ACID Properties − DBMS follows the concepts of Atomicity, Consistency, Isolation, and
Durability (normally shortened as ACID). These concepts are applied on transactions,
which manipulate data in a database. ACID properties help the database to stay healthy in
multi-transactional environments and also in case of failure.}
\item{Security − Features like multiple views offer security to certain extent when users are
unable to access the data of other users and departments. DBMS offers methods to impose
constraints while entering data into the database and retrieving the same at a later stage.
DBMS offers many different levels of security features, which enables multiple users to
have different views with different features. For example, a user in the Sales department
cannot see the data that belongs to the Purchase department. It can also be helpful in
deciding how much data of the Sales department should be displayed to the user. Since a
DBMS is not saved on the disk as traditional file systems, it is very hard for miscreants to
break the code.}
\item{Multiuser and Concurrent Access − DBMS supports multi-user environment and allows
them to access and manipulate data in parallel. Though there are restrictions on transactions
when users attempt to handle the same data item, but users are always unaware of them.
}
\end{itemize} 
\par 
\par
\section{APPLICATIONS OF DBMS}
Applications of Database Management Systems :
\begin{itemize}
\item {Telecom: There is a database to keeps track of the information regarding the calls made,
network usage, customer details etc. Without the database system it is hard to maintain such
huge amounts of data which gets updated every millisecond.}
\item{Industry: Whether it is a manufacturing unit, a warehouse or a distribution centre, each one
needs a database to keep the records of the ins and outs. For example, a distribution centre
should keep a track of the product units that were supplied to the centre as well as the
products that got delivered from the distribution centre on each day; this is where DBMS
comes into picture}
\item {Banking System: For storing information regarding a customer, keeping a track of his/her
day to day credit and debit transactions, generating bank statements etc is done with through
Database management systems.}
\item {Education sector: Database systems are frequently used in schools and colleges to store and
retrieve the data regarding the student , staff details, course details, exam details, payroll
data, attendance details, fees details etc. There is lots of inter-related data that needs to be
stored and retrieved in an efficient manner.}
\item{Online shopping: You must be aware of the online shopping websites such as Amazon, Flip
kart etc. These sites store the product information, your addresses and preferences, credit
details and provide you the relevant list of products based on your query. All this involves a
Database management system.}
\end{itemize}
\par
\section{PROBLEM DESCRIPTION/STATEMENT}
Placement Experience Database System deals with the information of the candidates who have been recruited by the corporate companies.
It involves following functionalities:
\begin{itemize}
    \item It consists of Student,Companies and Info.
    \item Each student can Login or Sign up according to their requirement
    \item He/She can view their status in each of the following companies.
\end{itemize}