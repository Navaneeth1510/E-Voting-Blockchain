\chapter{Requirement Analysis}
\thispagestyle{special}
\section{ Hardware Requirements }
The Hardware requirements are very minimal and the program can be run on most of the machines.
Processor : i5 processor
Processor Speed : 1.2 GHz
RAM : 1 GB
Storage Space : 40 GB
Monitor Resolution : 1024*768 or 1336*768 or 1280*1024
\section{ Software Requirements}
1. Operating System used: Windows 10
2. Technologies used: HTML, CSS, PHP, Bootstrap
3. XAMPP Server: MySQL, PhpMyAdmin
4. IDE used: Visual Studio Code
5. Browser that supports HTML
\section{Functional Requirements}
\subsection{Major Entities}
\subsection{End User Requirements}
\subsection{HTML}
Hypertext Markup Language (HTML) is the standard markup language for creating web pages and
web applications. With Cascading Style Sheets (CSS) and JavaScript it forms a triad of cornerstone
technologies for the World Wide Web. Web browsers receive HTML documents from a web server
or from a local storage and render them to multimedia web pages. HTML describes the structure of a
web page semantically and originally included cues for the appearance of the document.
HTML elements are the building blocks of HTML pages. With HTML constructs, images and
other objects like interactive forms can be embedded into the rendered page. It provides a way to
create structured documents by denoting structural semantics for the text like headings, paragraphs,
lists, links, quotes and other items. HTML elements are delimited by tags that are written within
angle brackets. Tags such as img tag and input tag introduce content into the page directly. Other tags
such as ¡p¿...¡/p¿ surround and provide information about document text and may include other tags
as sub-elements. Browsers do not display the HTML tags, but use them to interpret the content of the
page.
HTML can also embed programs written in a scripting language such as JavaScript which affect
the behaviour and content of web pages. Inclusion of CSS defines the look and layout of content.
\subsection{CSS}
Cascading Style Sheets (CSS) is a style sheet language which is used for describing the presentation
of a document written in markup language. Although most often its used to set the visual style of
web pages and user interfaces written in HTML and XHTML, the language can be applied to any
XML document, including plain XML, SVG and XUL, and is also applicable to rendering in speech,
or on other media. Along with HTML and JavaScript, CSS is a cornerstone technology used by
most websites to create visually engaging webpages, user interfaces for web applications, and user
interfaces for many mobile applications.
CSS is designed primarily to enable the separation of presentation and content, including aspects
such as the layout, colours, and fonts. This separation can improve content accessibility, provide more
flexibility and control in the specification of presentation characteristics, enable multiple HTML pages to share the formatting by specifying the relevant CSS in a separate .css file, and reduce complexity
and repetition in the structural content.
\subsection{PHP}
PHP is a server-side scripting language designed primarily for web development but is also used as
a general-purpose programming language. Originally created by Rasmus Lerdorf in 1994, the PHP
reference implementation is now produced by The PHP Development Team. PHP originally stood for
Personal Home Page, but it now stands for the recursive acronym PHP: Hypertext Pre-processor.
PHP code can be embedded into HTML or HTML5 markup, or it can be used in combination
with various web template systems, web content management systems and web frameworks. PHP
code is usually processed by a PHP interpreter implemented as a module in the web server or as
a Common Gateway Interface (CGI) executable. The web server software combines the results of
the interpreted and executed PHP code, which may be any type of data, including images, with the
generated web page.PHP code can also be executed with a command-line interface (CLI) and can be
used to implement standalone graphical applications.
The standard PHP interpreter, powered by the Zend Engine, is a free software released under the
PHP License. PHP has been widely ported and can be deployed on most web servers, on almost every
operating system and platform, free of charge. The PHP language evolved without a written formal
specification or standard until 2014, leaving the canonical PHP interpreter as a de facto standard.
Since 2014 work has gone into creating a formal PHP specification. HP development began in 1995
when Rasmus Lerdorf wrote several Common Gateway Interface (CGI) programs in C, which he used
in order to maintain Restaurant Management System his personal homepage. He extended them to
work with web forms and to communicate with databases, and called this implementation ”Personal
Home Page/Forms Interpreter” or PHP/FI
PHP/FI could help to build simple, dynamic web applications. To accelerate bug reporting and to
improve the code, Lerdorf initially announced the release of PHP/FI as ”Personal Home Page Tools
(PHP Tools) version 1.0” on the Usenet discussion group on June 8, 1995 This release already had
the basic functionality that PHP has as of 2013. This included Perl-like variables, form handling, and
the ability to embed HTML. The syntax resembled that of Perl but was simpler, more limited and less
consistent.
\subsection{MySQL}
MySQL is a Relational Database Management System (RDBMS). MySQL server can manage many
databases at the same time. In fact, many people might have different databases managed by a single
MySQL server. Each database consists of a structure to hold onto the data itself. A data-base can
exist without data, only a structure, be totally empty, twiddling its thumbs and waiting for data to be
stored in it. Data in a database is stored in one or more tables. You must create the data-base and
the tables before you can add any data to the database. First you create the empty database. Then
you add empty tables to the database. Database tables are organized in rows and columns. Each row
represents an entity in the database, such as a customer, a book, or a project. Each column contains
an item of information about the entity, such as a customer name, a book name, or a project start date.
The place where a particular row and column intersect, the individual cell of the table, is called a
field. Tables in databases can be related. Often a row in one table is related to several rows in another
table. For instance, you might have a database containing data about books you own. You would have
a book table and an author table. One row in the author table might contain information about the
author of several books in the book table. When tables are related, you include a column in one table
to hold data that matches data in the column of another table.
MySQL, the most popular Open Source SQL database management system, is developed, distributed, and supported by MySQL AB. MySQL 
AB is a commercial company, founded by the
MySQL developers. It is a second generation Open Source company that unites Open Restaurant
Management System Source values and methodology with a successful business model.
\begin{itemize}
\item{MySQL is a database management system. A database is a structured collection of data. It can
be anything from a simple shopping list to a picture gallery or the vast amount of information
in a corporate network. To add, access, and process data stored in a computer database, you
need a database management system such as MySQL Server. Since computers are very good at
handling large amounts of data, database management systems play a central role in computing,
as standalone utilities, or as parts of other applications.}
\item{ MySQL is a relational database management system. A relational database stores data in
separate tables rather than putting all the data in one big storeroom. This adds speed and
flexibility. The SQL part of “MySQL” stands for “Structured Query Language.” SQL is the
most common standardized language used to access databases and is defined by the ANSI/ISO
SQL Standard. The SQL standard has been evolving since 1986 and several versions exist.
“SQL-92” refers to the standard released in 1992, “SQL:1999” refers tothe standard released in 1999, and “SQL:2003” refers to the current version of the standard. We use the phrase “the
SQLstandard” to refer to the current version of the SQL Standard.
}
\item{MySQL software is Open Source. Open Source means that it is possible for anyone to use and
modify the software. Anybody can download the MySQL software from the Internet and use it
without paying anything. If you wish, you may study the source code and change it to suit your
needs. The MySQL software uses the GPL (GNU General Public License), to define what you
may and may not do with the software in different situations. The MySQL Database Server is
very fast, reliable, and easy to use.}
\end{itemize}
MySQL Server was originally developed to handle large databases and has been successfully used in
highly demanding production environments for several years. MySQL Server today offers a rich and
useful set of functions. Its connectivity, speed, and security make MySQL Server highly suited for
accessing databases on the Internet.
\subsection{XAMPP Server}
Xampp server installs a complete, ready-to-use development environment. Xampp server allows
you to fit your needs and allows you to setup a local server with the same characteristics as your
production.
While setting up the server and PHP on your own, you have two choices for the method of connecting PHP to the server. For many servers, PHP has a direct module interface (also called SAPI).
These servers include Apache, Microsoft Internet Information Server, Netscape and iPlanet servers.
Many other servers support ISAPI, the Microsoft module interface (OmniHTTPd for example). If
PHP has no module support for your web server, you can always use it as a CGI or FastCGI processor.
This means you set up yourserver to use the CGI executable of PHP to process all PHP file requests
on the server.
\begin{itemize}
\item{OpenGL is case sensitive}
\item{Line Color, Filled Faces and Fill Color not supported.}
\item{Shadow plane is not supported.}
\end{itemize}



%%********************Chapter 3**********
